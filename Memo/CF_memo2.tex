%%%%%%%%%%%%%%%%%%%%%%%%%%%%%%%%%%%%%%%%%
% Canadian Forces Memo
% LaTeX Template
% Version 1.0 (21/11/15)
%
% This template has been downloaded from:
% http://www.LaTeXTemplates.com
%
% Original author:
% K. Thomas
% With extensive modifications by:
% Vel (vel@latextemplates.com)
%
% Template license:
% CC BY-NC-SA 3.0 (http://creativecommons.org/licenses/by-nc-sa/3.0/)
%
% Important notes:
% 1) Common LaTeX distributions such as MikTeX, MacTeX and TeX Live are
% open source software and therefore are not authorized on DND computers. You
% therefore have install and typeset your memo on personal computers at home to 
% produce a pdf file of your memo that you transfer via email or protected USB stick.
% 2) This template was created from the American Math Society Article class with tweaks.
%
%%%%%%%%%%%%%%%%%%%%%%%%%%%%%%%%%%%%%%%%%

%----------------------------------------------------------------------------------------
%	PACKAGES AND OTHER DOCUMENT CONFIGURATIONS
%----------------------------------------------------------------------------------------

\documentclass[12pt]{amsart} % Using the American Math Society Article class set to 12pt font

\usepackage[letterpaper]{geometry} % Use US letter size paper

\setlength\parindent{0pt} % Stop paragraph indentation

\usepackage{mathptmx} % Use the Times New Roman font as suggested by the CF writing guide

\usepackage[normalem]{ulem} % Used for underlining text
\usepackage{hyperref}

\thispagestyle{empty} % Suppress headers, footers and page numbers on the first page
\pagestyle{empty} % Suppress headers, footers and page numbers on all pages

%----------------------------------------------------------------------------------------
%	MARGIN CONFIGURATIONS
%----------------------------------------------------------------------------------------

\addtolength{\oddsidemargin}{4pt} % This is the oddside margin used in oneside articles; 4 pts will add the 1/16th of an inch to get the printing to one inch from the edge of the paper

\addtolength{\textwidth}{40pt} % This command adds width to the text to get it to print 6.5 inches wide (470pts)

\addtolength{\topmargin}{-4pt} % This command will pull the top up the needed 1/32 inch to get it to print exactly 1 inch from top of paper

\addtolength{\textheight}{94pt} % This command enables the text to cover 9 inches (650pts) height leaving 1 inch on the top and bottom for the margins.

% Other margin tweaks to meet CF requirements
\addtolength{\marginparsep}{-11pt}
\addtolength{\marginparwidth}{-113pt}
\addtolength{\evensidemargin}{-38pt}
\addtolength{\headsep}{-14pt}
\addtolength{\headheight}{-8pt}
\addtolength{\footskip}{-12pt}
\addtolength{\marginparpush}{-5pt}

%----------------------------------------------------------------------------------------

\begin{document}

%----------------------------------------------------------------------------------------
%	MEMO HEADER
%----------------------------------------------------------------------------------------

% The military writing guide says that as soon as a Service Number is included on a document that at least PROTECTED A be on the document
% If accompanying documents are PROTECTED B then the memo should also be B
% If this is not the case comment out the PROTECTED A or B lines with a preceding % sign
% LaTeX will take care of moving the Memorandum line up to the margin automatically
\uline{\normalsize{INTRODUCCIÓN A  MÉTODOS NUMÉRICOS PARA FÍSICA TEÓRICA}}% Change this from B to A as needed

\vspace{14pt}

\large{Semestre: enero-junio 2025}

\vspace{14pt}

%------------------------------------------------

\textbf{Impartido por:} Armando A. Roque Estrada.
\vspace{8pt}

Para inscribirse en el curso, enviar un correo con sus datos a la dirección al calce del documento. Como requisitos se ha de estar cursando mínimo el 5to semestre de licenciatura.

\vspace{8pt}

%\hspace{0.375in}  Horario del curso (pendiente):
%\begin{table}[h]
%	\begin{tabular}{c| c| c| l}
%		& Inicio & Término& Local \\
%		\hline
%		- & -  & - & -\\
%		- & -  & -&  -\\
%		- & - & - &  - \\\hline
%	\end{tabular}
%\end{table}
%\vspace{8pt}

\textbf{Descripción del curso:}
 \vspace{8pt}
 
Las clases combinarán indistintamente conferencias y clases prácticas (implementación de códigos numéricos). A continuación, se presentan algunos detalles del curso:
\vspace{8pt}

Se abordarán las bases conceptuales (no se profundizará en demasía) de los aspectos teóricos que se implementarán en el cursos . Se cubrirá un conjunto de tópicos que permitirá profundizar en conocimientos de programación y matemáticas, además de servir como una herramienta fundamental para futuros trabajos científicos.
\vspace{8pt}

%El objetivo será desarrollar una comprensión conceptual básica de los temas centrales y familiarizarse con la parte práctica (implementación numérica). Un punto importante es que en este curso se introducirán diferentes conceptos que les permitirán comprender códigos numéricos ya establecidos.
%\vspace{8pt}

Como lenguaje principal, usaremos Python; sin embargo, en casos puntuales, podremos utilizar otros lenguajes. El curso se dividirá en dos bloques principales y un bloque de introducción a Python. Toda la información se puede encontrar en el repositorio: \href{https://github.com/Mandy8808/Metodos_Numericos_2024.git}{Github}

\vspace{8pt}

\textbf{Bloque Cero} (El ABC de Python)
\begin{enumerate}
	\item[-] Buenas prácticas al programar.
	\item[-] Ideas básicas sobre Python y algunos paquetes fundamentales.
\end{enumerate}

\textbf{Primer Bloque:}
\begin{itemize}
	\item[-] Números.
	\item[-] Derivadas.
	\item[-] Matrices
	\item[-] Roots.
\end{itemize}

\textbf{Segundo Bloque:}
\begin{itemize}
	\item[-] Interpolación.
	\item[-] Fitting.
	\item[-] Integrales.
	\item[-] Ecuaciones diferenciales ordinarias en el contexto de equilibrio hidrostático.
	\item[-] Ecuaciones diferenciales parciales. (tentativo)
	\item[-] Redes neuronales. (tentativo)
\end{itemize}

\vspace{14pt}

Para cualquier duda pueden contactarme por mi correo personal:

\vspace{4pt}

\textbf{arestrada@fisica.uaz.edu.mx} 

\vspace*{22pt} % This is intended to be used for the physical space for signature with a pen

\textbf{Fuentes de Información}
\vspace{10pt}
\begin{itemize}
	\item[1.] Numerical Methods in Physics with Python, Alex Gezerlis. (2020, Cambridge University Press)
	\item[2.] Python Programming and Numerical Methods A Guide for Engineers and Scientist, Qingkai Kong, Timmy Siauw, Alexandre Bayen.
\end{itemize}

%------------------------------------------------

\vspace*{22pt}

\uline{\large{\textbf{Acondicionamiento del equipo de cómputo}}}

\vspace*{14pt}

En caso de querer instalar  \textbf{Python 3.x} a continuación se brindan unos consejos para su instalación:
\vspace{14pt}

%------------------------------------------------

\hspace{0.375in} Los consejos que se comentan a continuación sirven de guía para instalar en Linux, Mac OS o Windows mediante el software (gestor de paquetes)  Anaconda. Una alternativa puede ser descargar directamente Python: \url{https://www.python.org} y  el gestor de paquetes PIP (ver tutorial \url{https://acortar.link/RG5PS8}).

\vspace{12pt}

Pasos:
\begin{itemize}
	\item[1-] Descargar e instalar la versión de Anaconda correspondiente al sistema o\-perativo y versión de Python usada (se recomienda python 3.)
	 
	\url{https://www.anaconda.com/products/individual}
	
	\item[3-] Configurar Anaconda,
	
	\begin{enumerate}
		\item Actualizar \textit{conda} y \textit{jupyter}, abrir una terminal y teclear,
			\vspace{8pt}
	
			\emph{conda update conda}
	
			\emph{conda update jupyter}
		\item Instalar librerías básicas,
			\vspace{8pt}
	
			\emph{conda install anaconda::numpy} (\url{https://numpy.org})
	
			\emph{conda install conda-forge::matplotlib} (\url{https://matplotlib.org})
	
			\emph{conda install anaconda::scipy} (\url{https://scipy.org})
	
		\emph{conda install anaconda::pandas} (\url{https://pandas.pydata.org})
	
	\vspace{8pt}
	
	\textbf{Otras:} TensorFlow, PyTorch, Keras, Scikit-learn, Seaborn, Bokeh, Sympy, Numba
	\item Comprobar la instalación, escribir en una terminal y teclear
	\vspace{8pt}
	\emph{jupyter-notebook}
		\end{enumerate}
\end{itemize}

Posibles errores:
\vspace{8pt}

No se ejecuta, python, conda o jupyter-notebook desde la consola. Posible solución:

\vspace{8pt}
Windows

\url{https://acortar.link/RHLKrH}

\vspace{8pt}
MacOS y Linux

Es necesario modificar el .bash\_profile, contactarme por correo para explicarles como hacerlo. 

\vspace{14pt}

En caso de no querer instalar Python en su equipo de cómputo, pueden hacer uso de la plataforma 
\\~\\
\textbf{Colaboratory}: \url{https://colab.research.google.com/?hl=es}
\\~\\
la cual cuenta con la mayoría de los recursos necesarios para un buen desarrollo del curso. En necesario señalar que han de tener una cuenta en gmail para su uso.


\vspace{14pt}

\parbox[c]{235pt}{EDITOR}
\vspace{8pt}

Una de las herramientas más útiles para programar son los editores, los mismos nos ahorran mucho trabajo. Aunque para el caso de Python podemos usar (y usaremos) la herramienta jupyter, pero cuando se trabaja con servidores, etc. es necesario crear \textit{scripts} con nuestros códigos, siendo los editores la herramienta ideal para ello.  Uno que recomiendo es Visual Studio Code \url{https://code.visualstudio.com} con las extensiones: Python, Jupyter, Python Indent, Python Extended.

%----------------------------------------------------------------------------------------

\end{document}
